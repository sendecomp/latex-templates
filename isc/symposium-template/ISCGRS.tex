\documentclass{ISCGRS}

\copyrightnotice{\copyright Ned Flanders 2014}

\conferenceheader{Proceedings of the 8th Annual ISC Graduate Research Symposium \\
ISC-GRS 2014 \\
April 22, 2014, Rolla, Missouri}

\title{Put Paper Title Here}

\author{\IEEEauthorblockN{Michael Shell}
\IEEEauthorblockA{School of Electrical and\\Computer Engineering\\
Georgia Institute of Technology\\
Atlanta, Georgia 30332--0250\\
Email: http://www.michaelshell.org/contact.html}
\and
\IEEEauthorblockN{Homer Simpson}
\IEEEauthorblockA{Twentieth Century Fox\\
Springfield, USA\\
Email: homer@thesimpsons.com}
\and
\IEEEauthorblockN{James Kirk\\ and Montgomery Scott}
\IEEEauthorblockA{Starfleet Academy\\
San Francisco, California 96678-2391\\
Telephone: (800) 555--1212\\
Fax: (888) 555--1212}}

\begin{document}
\maketitle

\begin{abstract}
Put abstract text here. The first line of the abstract text should not be indented. The text font should be Times New Roman.
\end{abstract}

\section {Introduction}
Place the introduction to your paper here. The first paragraph is not indented. Subsequent paragraphs are indented by 0.25 inches.

\section {First Body Section}
Put body of the paper here. The first paragraph is not indented. All subsequent paragraphs are indented by 0.25 inches.

The first line of text of the second and subsequent paragraphs in the body text is indented by 0.25 inches.

\subsection{Body Section}
Text for body section 2.1 goes here. The first line of the text is indented by 0.25 inches.

\subsection{Body Section}
Text for body section 2.1 goes here. The first line of the text is indented by 0.25 inches.

\section{TABULATIONS AND ENUMERATIONS}
Where several considerations, conditions, requirements, or other qualifying items are involved in a presentation, it is often advantageous to put them in tabular or enumerative form, one after the other, rather than to run them into the text. This arrangement, in addition to emphasizing the items, creates a graphic impression that aids the reader in accessing the information and in forming an overall picture. It is customary to identify the individual items as (1), (2), (3), etc., or as (a), (b), (c), etc. Although inclusion of such elements makes the text livelier, care should be taken not to use this scheme too frequently, as it can make the reading choppy and invalidate their purpose and usefulness.

\section { FIGURES AND TABLES}

\subsection{Figures}
All figures (graphs, line drawings, photographs, etc.) should be numbered consecutively and have a caption consisting of the figure number and a brief title or description of the figure. This number should be used when referring to the figure in text. Figures should be referenced within the text as "Fig. 1." When the reference to a figure begins a sentence, the abbreviation "Fig." should be spelled out, e.g., "Figure 1."
Figures may be inserted as part of the text, or included on a separate page immediately following or as close as possible to its first reference — with the exception of those figures included at the end of the paper as an appendix.
All graphs, line drawings, photographs, etc., must be clear and sharp and of best available quality.

\subsection{Cropping and Orientation}
Graphics should be cropped to remove any unnecessary white space around the image, and should not include borders.
Image orientation should be the same as intended for publication.

3 9/16 in. - to fit width of one column

7 1/2 in. - to fit across the top of the page (span two columns)

9 x 7 1/2 in. (L x W) - to fit the entire page (top to bottom)

6 1/2 in. x 81/2 in. (L x W) - across the entire page sideways (turned)

\subsection{Color Graphics}
The publication of the paper will be in digital form, and color graphics will be accepted in the digital publication.

\subsection{Fonts}
The appropriate font(s) should be used when labeling your figures.

Font Family: True Type (Arial, Helvetica, Times New Roman, Courier) or Adobe fonts

Font Format: embedded in graphic

Font Size: 6 pt minimum

\subsection{Miscellaneous}
Line weights should be 0.5 pt - 1.5 pt in thickness (line weights below 0.5 points will reproduce poorly). Providing graphics at a lower resolution than required, or at a size less than 100\% of final, will result in jagged lines, pixilated type, and an unacceptable final image. Providing graphics at a higher resolution than required, or at a size above 100\% of final, will result in unnecessarily large files.

\subsection{Tables}
All tables should be numbered consecutively and have a caption consisting of the table number and a brief title. This number should be used when referring to the table in text. Tables may be inserted as part of the text, or included on a separate page immediately following or as close as possible to its first reference — with the exception of those tables included at the end of the paper as an appendix.

\section{MATHEMATICS}
Display equations should be set apart from the body of the text and centered. Use two or three line spaces to separate equations from text. Equations should be numbered consecutively beginning with (1) to the end of the paper, including any appendices. The number should be enclosed in parentheses (as shown above) and set flush right in the column on the same line as the equation. No ellipses (dots) from the equation to the equation number, or any punctuation at the end of the equation itself. It is this number that should be used when referring to equations within the text. Equations should be referenced within the text as "Eq. (x)." When the reference to an equation begins a sentence, it should be spelled out, e.g., "Equation (x)."

Formulas and equations should be created to clearly distinguish capital letters from lowercase letters. Care should be taken to avoid confusion between the lowercase "l'' (el) and the numeral one, or between zero and the lowercase "o.'' All subscripts, superscripts, Greek letters, and other symbols should be clearly indicated.

In all mathematical expressions and analyses, any symbols (and the units in which they are measured) not previously defined in nomenclature should be explained. If the paper is highly mathematical in nature, it may be advisable to develop equations and formulas in appendices rather than in the body of the paper.

\section{PAGE NUMBER AND FOOTER}
All conference papers should be numbered. Page number should be centered at the bottom of each page. A copyright footer should also be included in the bottom right-hand corner of each page.

\section{CONCLUSIONS}
Put paper conclusions here

\section{ACKNOWLEDGMENTS}
Put acknowledgments here. Acknowledgements may be made to individuals or institutions not mentioned elsewhere in the paper, who have made an important contribution. You must acknowledge the support of the Intelligent Systems Center for your research presented in the paper.

\section{REFERENCES}
Within the text, references should be cited in numerical order according to their order of appearance. The numbered reference citation should be enclosed in brackets. In the case of two citations, the numbers should be separated by a comma [1,2]. In the case of more than two reference citations, the numbers should be separated by a dash [5-7].

References to original sources for cited material should be listed together at the end of the paper; footnotes should not be used for this purpose. References should be arranged in numerical order according to their order of appearance within the text.

\begin{thebibliography}{1}

\bibitem{IEEEhowto:kopka}
H.~Kopka and P.~W. Daly, \emph{A Guide to \LaTeX}, 3rd~ed.\hskip 1em plus
  0.5em minus 0.4em\relax Harlow, England: Addison-Wesley, 1999.

\end{thebibliography}


\end{document} 